\chapter{Validação}\label{chap:chap4}

\section*{}

De forma a validar que os objetivos que propomos são atingidos serão utilizados como exemplo projetos com as seguintes características:
%
\begin{itemize}
	\item Usem o sistema de controlo de versões \emph{Git}.
	\item Tenham mensagem descritivas associadas às alterações (\emph{commits}), permitindo a identificação das correções.
	\item Sejam compatíveis com o \emph{software Crowbar}.
	\item Tenham um número elevado de testes automáticos \emph{JUnit}.
	\item Tenham defeitos previamente identificados.
\end{itemize}

Pondera-se o uso de coleções de projetos criadas previamente para o efeito de validação de resultados na área de \emph{software repository mining} e \emph{fault localization software}, tal como Defects4J. % TODO: Cite

Através da análise destes projetos será validado se é atribuído aos componentes com defeitos um valor de \emph{component goodness} tendencialmente mais baixo e se este é capaz de contribuir positivamente para a ordenação de componentes feita pelo \emph{Barinel}. Considera-se melhor ordenação aquela que apresente os componentes defeituosos mais próximos do topo da lista.