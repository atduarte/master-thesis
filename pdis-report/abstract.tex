\chapter*{Resumo}

O Resumo fornece ao leitor um sumário do conteúdo da dissertação.
Deverá ser breve mas conter detalhe suficiente e, uma vez que é a porta
de entrada para a dissertação, deverá dar ao leitor uma boa impressão
inicial.

Este texto inicial da dissertação é escrito no fim e resume numa
página, sem referências externas, o tema e o contexto do trabalho, a
motivação e os objectivos, as metodologias e técnicas empregues, os
principais resultados alcançados e as conclusões.

Este documento ilustra o formato a usar em dissertações na \Feup.
São dados exemplos de margens, cabeçalhos, títulos, paginação, estilos
de índices, etc. 
São ainda dados exemplos de formatação de citações, figuras e tabelas,
equações, referências cruzadas, lista de referências e índices.
%Este documento não pretende exemplificar conteúdos a usar. 
É usado texto descartável, \emph{Loren Ipsum}, para preencher a
dissertação por forma a ilustrar os formatos.

Seguem-se umas notas breves mas muito importantes sobre a versão 
provisória e a versão final do documento. 
A versão provisória, depois de verificada pelo orientador e de 
corrigida em contexto pelo autor, deve ser publicada na página 
pessoal de cada estudante/dissertação, juntamente com os dois 
resumos, em português e em inglês; deve manter a marca da água, 
assim como a numeração de linhas conforme aqui se demonstra.

A versão definitiva, a produzir somente após a defesa, em versão 
impressa (dois exemplares com capas próprias FEUP) e em versão 
eletrónica (6 CDs com "rodela" própria FEUP), deve ser limpa da marca de 
água e da numeração de linhas e deve conter a identificação, na primeira 
página, dos elementos do júri respetivo. 
Deve ainda, se for o caso, ser corrigida de acordo com as instruções 
recebidas dos elementos júri.

Lorem ipsum dolor sit amet, consectetuer adipiscing elit. Sed vehicula
lorem commodo dui. Fusce mollis feugiat elit. Cum sociis natoque
penatibus et magnis dis parturient montes, nascetur ridiculus
mus. Donec eu quam. Aenean consectetuer odio quis nisi. Fusce molestie
metus sed neque. Praesent nulla. Donec quis urna. Pellentesque
hendrerit vulputate nunc. Donec id eros et leo ullamcorper
placerat. Curabitur aliquam tellus et diam. 

Ut tortor. Morbi eget elit. Maecenas nec risus. Sed ultricies. Sed
scelerisque libero faucibus sem. Nullam molestie leo quis
tellus. Donec ipsum. Nulla lobortis purus pharetra turpis. Nulla
laoreet, arcu nec hendrerit vulputate, tortor elit eleifend turpis, et
aliquam leo metus in dolor. Praesent sed nulla. Mauris ac augue. Cras
ac orci. Etiam sed urna eget nulla sodales venenatis. Donec faucibus
ante eget dui. Nam magna. Suspendisse sollicitudin est et mi. 

Phasellus ullamcorper justo id risus. Nunc in leo. Mauris auctor
lectus vitae est lacinia egestas. Nulla faucibus erat sit amet lectus
varius semper. Praesent ultrices vehicula orci. 

Ut tortor. Morbi eget elit. Maecenas nec risus. Sed ultricies. Sed
scelerisque libero faucibus sem. Nullam molestie leo quis
tellus. Donec ipsum. 

\vspace*{10mm}\noindent
\textbf{Keywords}: keyword1, Keyword2, keyword