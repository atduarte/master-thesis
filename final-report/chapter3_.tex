\chapter{Perspetiva de Solução}\label{chap:chap3}

\section*{}

Com base nas informações recolhidas durante a revisão bibliográfica e algum estudo feito entretanto, temos já uma perspetiva de solução que deverá permitir atingir os objetivos propostos.

O objetivo será melhorar a ordenação do \emph{Barinel}, através do cálculo de valores que representem a probabilidade de cada um dos componentes não ter erro, a partir dos dados retirados do repositório \emph{Git} do projeto. Este valor substituirá ou $g_j$, a \emph{component goodness}, ou $p_j$, a \emph{prior} do determinado componente.
%
Sendo que este cálculo se trata de uma previsão baseada no histórico do projeto e em que se pretende classificar cada componente como mais ou menos propenso a erros, percebemos que se trata de um problema de classificação binomial (\emph{Machine Learning}), em que as \emph{labels} serão \emph{buggy} (com defeitos) ou \emph{clean}. Sendo que, caso o modelo seja correto, a $confidence_j(clean)$ corresponderá ao valor que precisamos.

Baseando-nos nos resultados obtidos pelo \emph{Change Classification}, achamos pertinente começar por identificar as alterações (\emph{commits}) que correspondem a correções, através da mensagem descritiva associada.
A partir desta identificação, pretendemos extrair as informações relativas a cada um dos componentes no momento anterior à sua correção. As informações a extrair terão ainda de ser elegidas, mas deverão envolver os seguintes dados:
%
\begin{itemize}
	\item Existência de erros (considerar-se-á que os componentes alterados durante a correção correspondem aos componentes defeituosos).
	\item Data de alterações anteriormente.
	\item \emph{Usernames} dos últimos contribuidores.
	\item Complexidade e número de linhas de código do componente.
	\item Existência de comentários no código indicando necessidade de trabalhos futuros.
\end{itemize}

Após esta extração de informações do \emph{Git}, estes dados terão de ser processados para permitir a criação de um modelo de classificação que seja capaz de representar os componentes da forma que pretendemos. Um dos processamentos que possivelmente terá de ser efetuado é relativo às datas de alterações, pois pretende-se um valor que represente se o componente tem vindo ou não a ser modificado diversas vezes. Para este exemplo considera-se o uso de uma função que considera, através da atribuição de pesos, como mais relevante as alterações que tenham uma data mais próxima da sua correção.

Tendo os dados sido devidamente extraídos e processados para cada componente de cada correção feita durante o decorrer do projeto, tenciona-se usar um dos seguintes algoritmos de \emph{Machine Learning}:
%
\begin{itemize}
	\item \emph{Naïve Bayes}.
	\item \emph{Support Vector Machines}.
	\item \emph{Random Forests}.
\end{itemize}

Obtendo no final para cada componente o valor que necessitávamos, a $confidence_j(clean)$. Como dito anteriormente, este valor deverá representar a probabilidade do componente não ter erros e será utilizado pelo \emph{Barinel/Crowbar} no cálculo da probabilidade de cada candidato. A experimentação será necessária para determinar se esta probabilidade será mais útil para a ordenação dos resultados substituindo $g_j$ (\emph{component goodnesses}) ou $p_j$ (\emph{priors}).

\subsection{Possíveis dificuldades}

Identificamos à partida algumas dificuldades e problemas que poderão surgir durante a realização do trabalho:
%
\begin{itemize}
	\item Poderá não ser possível em tempo razoável conseguir extrair do repositório \emph{Git} toda a informação de que necessitamos.
	\item Em alguns projetos a identificação das correções pode não ser muito precisa, devido a mensagens pouco descritivas.
	\item Classificar todos os componentes que não foram alterados na correção como \emph{clean} poderá não ser o mais correto e afetar negativamente a criação do modelo. Isto porque poderão haver componentes que não foram modificados na altura, mas que mesmo assim contém erros ainda não corrigidos ou identificados.
	\item Havendo um número muito superior de componentes à partida classificados como \emph{clean} (sem erros) poderá haver a tendência do modelo classificar erradamente componentes \emph{buggy}.
	\item Poderá ser demonstrado que mesmo obtendo um valor de \emph{component goodness} representativo, o algoritmo \emph{Barinel} apresente melhores resultados continuando a recorrer ao algoritmo de maximização MLE.
\end{itemize}
