%!TEX root = ../thesis.tex
\chapter{Introduction} \label{chap:intro}

\section*{}

With this chapter we aim to frame the subject, explain our motivation and describe the document structure.

\section{Context} \label{sec:context}

With the big growth of the software development industry, the existence of tools that help programmers do a better job becomes even more important.

It is estimated that the economy of the United States of America loses 60 billion dollars every year in costs associated with the development and distribution of fixes to defects on software and in its re-installation \cite{Zhivich2009}. Given this, we can claim that tools of software fault localization, by helping reduce the time invested in this task, may have a huge positive impact in the economy. In the domain of software fault localization the research is considerable. \emph{Ochiai}, \emph{Tarantula}, \emph{Bayes-A} and Barinel are just some of the existing solutions, being the Barinel algorithm the one that reveals better results.
Even though, this algorithm can be optimized, returning more accurate results, by using defect probabilities estimated according to information related to the project.

Given that version control systems, such as \emph{Git}, contain all the project's history and information about all changes (e.g. changed content, date and authors), we believe this data, along with Machine Learning algorithms, may be the key to Barinel improvement.

\section{Motivation and Objectives} \label{sec:goals}

Given the possibilities of Machine Learning and information extraction from software repositories, with this dissertation we intend to:
%
\begin{itemize}
\item Be able to predict the defect probability of each component in a given software project that uses Git with an useful precision, regardless of the programming language used.
\item Optimize the order of results of Barinel.
\end{itemize}

\section{Structure} \label{sec:struct}

In addition to this chapter, this dissertation contains 6 more chapters.

Chapter \ref{chap:sota} describes the state of the art and presents work related to fault localization software, software repository mining and some other approaches to defect prediction. Chapter \ref{chap:estimating-dp} presents the concept and working details of the solution created to estimate the defect probability. Chapter \ref{chap:barinel-integration} explains both integrations tried, results modification and priors replacement. Chapters \ref{chap:exp-results} and \ref{chap:discussion} contain the experimental results obtained by experimentation and the discussion of those results, respectively. Chapter \ref{chap:conclusions} concludes the dissertation by analyzing the goals satisfaction, main contributions and further work.
