%!TEX root = ../thesis.tex
\section{\emph{Software Repository Mining}}

Entende-se por \emph{Software Repository Mining} todo o \emph{software} que é capaz objetivo de extrair dados de repositórios, como \emph{Git} ou \emph{SVN}.

\subsection{Identificação de correções e defeitos}

Na extração de informações de repositórios de código, como o \emph{Git}, torna-se importante identificar o tipo de alterações feitas. A descrição textual associada à alteração é bastante importante nesta identificação e permite fazê-lo com um elevado grau de precisão, dependendo do projeto em causa \cite{Mockus2000}.

É estimado que 34 a 46\% de todas as alterações correspondam a correções e que estas representem 18 a 27\% das linhas adicionadas e removidas \cite{Mockus2000}. Foram também encontrados padrões que relacionam as correções com o seu tamanho, em termos de código, e com o dia da semana em que foram executadas \cite{Sliwerski2005}.

Após esta identificação é possível identificar as alterações que introduziram os erros usando o algoritmo \emph{SZZ} \cite{Sliwerski2005}. Este algoritmo recorre ao histórico e à análise do código em questão para corretamente localizar o \emph{commit} (alteração) que introduziu o defeito, ignorando, por exemplo, alterações que apenas introduzam comentários.

\subsection{Ferramentas}

Existem diversas ferramentas que facilitam este processo, pelo que destacamos duas.

\subsubsection{\emph{libgit2}}

Libgit2\footnote{\url{https://libgit2.github.com/}} é uma biblioteca multi-plataforma, sem dependências, que permite a interação direta com o repositório \emph{Git} com uma performance nativa.

Esta biblioteca pode ser usada em qualquer linguagem que permita ligação a \emph{C}, pelo que existem dezenas de implementações em linguagem como \emph{Python}\footnote{pygit2 - \url{http://www.pygit2.org/}}, \emph{Javascript}\footnote{nodegit - \url{https://github.com/nodegit/nodegit}} e \emph{Java}\footnote{Jagged - \url{https://github.com/ethomson/jagged}}.

\subsubsection{\emph{GHTorrent}}

GHTorrent\footnote{\url{http://ghtorrent.org/}} é um projeto que tem por objetivo criar um armazém de dados escalável, facilmente pesquisável, mesmo \emph{offline}, que replique a informação obtida através da API REST do GitHub\footnote{\url{http://developer.github.com/}}.

Esta ferramenta funciona de uma forma distribuída e monitoriza a API do GitHub através da página \url{https://api.github.com/events}. Cada evento, despoleta a extração de informação e conteúdo para uma base de dados \emph{MongoDB} e outra \emph{MySQL} \cite{Gousios2012}.