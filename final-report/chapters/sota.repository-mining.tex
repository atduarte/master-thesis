%!TEX root = ../thesis.tex
\section{Software Repository Mining}

Software Repository Mining encompasses all software which is able to extract data from repositories, such as \emph{Git} or \emph{SVN}.

\subsection{Corrections and defects identification}

While extracting information from code repositories, such as \emph{Git}, it is important to identify the types of changes that were made. A text description associated with the change is of the utmost importance for this identification, allowing it to be made with a high degree of precision, depending on the project \cite{Mockus2000}.

It is estimated that 33.8\% of all bug reports are misclassified, by identifying as feature, documentation update or internal refactor, instead of correctly identifying it as code fix \cite{Herzig2013b}. Furthermore, on average 39\% of files marked as faulty actually never had a defect \cite{Herzig2013b}.

It is estimated that between 34 and 46\% of all changes are corrections and that they represent between 18 and 27\% of all added and removed lines of code \cite{Mockus2000}. There were also some patterns found relating code corrections with their size and with the day of the week in which they were executed \cite{Sliwerski2005}.

After this identification, it is possible to find out which changes have introduced the errors by using the \emph{SZZ} algorithm \cite{Sliwerski2005}. This algorithm analyzes the code and uses its history to correctly locate which \emph{commit} has introduced the defect, ignoring changes that only inserted comments, for example.

\subsection{Tools}

There are several tools to facilitate this process. Let us highlight two of them.

\subsubsection{\emph{Git-cli}}

There are a lot of different ways to extract information from Git, one of which is Git on the command line. This tool is multi-platform and allows to execute all Git commands.

Although, this tool does not have official bindings to any languages. So the only way of programmatically using it is by instantiating a \emph{git-cli} process and parse the output for each needed task.

There exists some \emph{Node.js} modules that abstract the instantiation and parsing, such as \emph{git-cli}\footnote{git-cli - \url{https://www.npmjs.com/package/git-cli}}.

\subsubsection{\emph{libgit2}}

Libgit2\footnote{\url{https://libgit2.github.com/}} is a multi-platform library, without dependences, allowing a direct and native performance interaction with the \emph{Git} repository.

This library can be used with any language supporting \emph{C} bindings, so there are dozens of implementations using languages like \emph{Python}\footnote{pygit2 - \url{http://www.pygit2.org/}}, \emph{Javascript}\footnote{nodegit - \url{https://github.com/nodegit/nodegit}} and \emph{Java}\footnote{Jagged - \url{https://github.com/ethomson/jagged}}.

\subsubsection{\emph{GHTorrent}}

GHTorrent\footnote{\url{http://ghtorrent.org/}} is a project with the objective of creating a scalable and easy to query (even \emph{offline}) database, which replicates the information obtained through the GitHub\footnote{\url{http://developer.github.com/}} REST API.

This tool works in a distributed way and monitors the GitHub API through the page \url{https://api.github.com/events}. Each event triggers an information and content extraction to a \emph{MongoDB} database and a different \emph{MySQL} database \cite{Gousios2012}.