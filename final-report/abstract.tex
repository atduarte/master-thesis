\chapter*{Abstract}
%\addcontentsline{toc}{chapter}{Abstract}

TODO

\vspace*{10mm}\noindent

\textbf{Keyworks}: \emph{Software-fault Localization}, \emph{Software Repository Mining}, \emph{Machine Learning}, \emph{Classification}

\vspace*{5mm}\noindent

\textbf{Classificação}: \emph{Software and its engineering - Software creation and management - Software verification and validation; Computing methodologies - Machine Learning - Machine Learning Approaches}

\chapter*{Resumo}
%\addcontentsline{toc}{chapter}{Resumo}

Dada a crescente necessidade de identificar a localização dos erros no código fonte de \emph{software}, de forma a facilitar o trabalho dos programadores e a acelerar o processo de desenvolvimento, muitos avanços têm sido feitos na sua automação.

Existem três abordagens principais: \emph{Program-spectra based} (PSB), \emph{Model-based diagnosis} (MDB) e \emph{Program slicing}.

\emph{Barinel}, solução que integra tanto o PSB como o MDB, é, até hoje, com base na investigação feita, a que apresenta melhores resultados. Contudo, a ordenação de conjuntos de candidatos (componentes faltosos) não tem em conta a verdadeira qualidade do componente em causa, mas sim o conjunto de valores que maximizam a probabilidade do conjunto (\emph{Maximum Likehood Estimation} - MLE), devido à dificuldade da sua determinação.

Com esta tese pretende-se colmatar esta falha e contribuir para uma melhor ordenação dos conjuntos, classificando, com recurso a técnicas de Machine Learning como \emph{Naive Bayes}, \emph{Support Vector Machines} (SVM) ou \emph{Random Forests}, a qualidade e fiabilidade de cada componente, através das informações disponíveis no sistema de controlo de versões (\emph{Software Repository Mining}), neste caso \emph{Git}, como por exemplo: número de vezes que foi modificado, número de contribuidores, data de última alteração, nome de últimos contribuidores e tamanho das alterações.

A investigação já feita, revelou a existência de algumas soluções de análise preditiva de \emph{software}, como \emph{BugCache}, \emph{FixCache} e \emph{Change Classification}, capazes de identificar componentes com grande probabilidade de falhar e de classificar as revisões (\emph{commits}) como faltosas ou não, mas nenhuma soluciona o problema.

Este trabalho visa também a integração com o \emph{Crowbar} e a contribuição para a sua possível comercialização.

\vspace*{10mm}\noindent

\textbf{Palavras-chave}: \emph{Software-fault Localization}, \emph{Software Repository Mining}, \emph{Machine Learning}, \emph{Classification}

\vspace*{5mm}\noindent

\textbf{Classificação}: \emph{Software and its engineering - Software creation and management - Software verification and validation; Computing methodologies - Machine Learning - Machine Learning Approaches}