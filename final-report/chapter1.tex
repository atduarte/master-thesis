\chapter{Introdução} \label{chap:intro}

\section*{}

Pretende-se com este capítulo enquadrar o tema, explicar as motivações e descrever sucintamente a estrutura deste documento.

\section{Contexto/Enquadramento} \label{sec:context}

Com o elevado crescimento da indústria de desenvolvimento de software,
torna-se cada vez mais importante a existência de ferramentas que auxiliem
os programadores a desenvolvê-lo mais eficientemente.

Estima-se que a economia dos Estados Unidos perca entre 60 mil milhões de dólares por ano em custos associados ao desenvolvimento e distribuição de correções para defeitos de \emph{software} e na sua reinstalação \cite{Zhivich2009}. Pelo que, podemos afirmar que as ferramentas de localização das falhas de \emph{software} (\emph{Software Fault Localization}), ajudando a reduzir o tempo investido nesta tarefa, poderão ter um impacto significativo na economia.
Nesta área os avanços são consideráveis. \emph{Ochiai}, \emph{Tarantula}, \emph{Bayes-A} e \emph{Barinel} são apenas algumas das soluções existentes, sendo o algoritmo \emph{Barinel} aquele que apresenta melhores resultados. \cite{Abreu2009}
Apesar dos bons resultados apresentados pelo \emph{Barinel}, este poderá apresentar resultados ainda mais rigorosos se tivermos informações relativas ao projeto, como a probabilidade média de erro ou a probabilidade de dado componente, que o constitui, ter defeitos.

Ferramentas de controlo de versões, como o \emph{Git}, uma vez que mantêm todo o histórico do projeto e informações relacionadas com as diversas alterações (p.e. conteúdo, data e autores), em conjunto com técnicas de \emph{Machine Learning}, poderão ser a chave para a melhoria deste algoritmo.

\section{Motivação e Objetivos} \label{sec:goals}

Tendo em conta as possibilidades que a extração de dados de repositórios de controlo de versões e o \emph{Machine Learning} nos dão, pretende-se com esta dissertação:
%
\begin{itemize}
\item Optimizar a ordenação de resultados candidatos do algoritmo Barinel.
\item Ter a capacidade de prever a probabilidade de defeito de cada um dos componentes de um dado projecto de \emph{software}, que use o \emph{Git} para controlo total de versões, com uma precisão útil.
\end{itemize}

\section{Estrutura} \label{sec:struct}

% TODO: CHANGE

% Para além da introdução, esta dissertação contém mais 4 capítulos.
% No capítulo~\ref{chap:sota}, é descrito o estado da arte e são apresentados trabalhos relacionados com \emph{fault localization software}, \emph{software repository mining} e ainda algumas abordagens à previsão de defeitos. No capítulo~\ref{chap:chap3}, expõe-se a perspetiva de solução e contrasta-se com as possíveis dificuldades que poderão ser encontradas durante a sua realização, apresentando no capítulo seguinte~\ref{chap:chap4} a forma como esta deve ser validada. Por último, o capítulo~\ref{chap:concl} conclui o relatório e explicita o plano de trabalhos, na forma de um gráfico de \emph{Gantt}.
